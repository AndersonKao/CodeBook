\documentclass[a4paper,10pt,oneside]{article}
\setlength{\columnsep}{15pt}    %兩欄模式的間距
\setlength{\columnseprule}{0pt}

\usepackage[landscape]{geometry}
\usepackage{amsthm}								%定義,例題
\usepackage{amssymb}
\usepackage{fontspec}								%設定字體
\usepackage{color}
\usepackage[x11names]{xcolor}
\usepackage{xeCJK}								%xeCJK
\usepackage{listings}								%顯示code用的
%\usepackage[Glenn]{fncychap}						%排版,頁面模板
\usepackage{fancyhdr}								%設定頁首頁尾
\usepackage{graphicx}								%Graphic
\usepackage{enumerate}
\usepackage{titlesec}
\usepackage{amsmath}
\usepackage{pdfpages}
\usepackage{multicol}
\usepackage{fancyhdr}
%\usepackage[T1]{fontenc}
\usepackage{amsmath, courier, listings, fancyhdr, graphicx}
\usepackage[smartEllipses]{markdown}

%\topmargin=0pt
%\headsep=5pt
\textheight=530pt
%\footskip=0pt
\voffset=-20pt
\textwidth=800pt
%\marginparsep=0pt
%\marginparwidth=0pt
%\marginparpush=0pt
%\oddsidemargin=0pt
%\evensidemargin=0pt
\hoffset=-100pt

%\setmainfont{Consolas}				%主要字型
\setCJKmainfont{msjh.ttc}			%中文字型
%\setmainfont{Linux Libertine G}
\setmonofont{consola.ttf}
%\setmainfont{sourcecodepro}
\XeTeXlinebreaklocale "zh"						%中文自動換行
\XeTeXlinebreakskip = 0pt plus 1pt				%設定段落之間的距離
\setcounter{secnumdepth}{3}						%目錄顯示第三層

\makeatletter
\lst@CCPutMacro\lst@ProcessOther {"2D}{\lst@ttfamily{-{}}{-{}}}
\@empty\z@\@empty
\makeatother
\lstset{											% Code顯示
language=C++,										% the language of the code
basicstyle=\scriptsize\ttfamily, 						% the size of the fonts that are used for the code
numbers=left,										% where to put the line-numbers
numberstyle=\tiny,						% the size of the fonts that are used for the line-numbers
stepnumber=1,										% the step between two line-numbers. If it's 1, each line  will be numbered
numbersep=5pt,										% how far the line-numbers are from the code
backgroundcolor=\color{white},					% choose the background color. You must add \usepackage{color}
showspaces=false,									% show spaces adding particular underscores
showstringspaces=false,							% underline spaces within strings
showtabs=false,									% show tabs within strings adding particular underscores
frame=false,											% adds a frame around the code
tabsize=2,											% sets default tabsize to 2 spaces
captionpos=b,										% sets the caption-position to bottom
breaklines=true,									% sets automatic line breaking
breakatwhitespace=false,							% sets if automatic breaks should only happen at whitespace
escapeinside={\%*}{*)},							% if you want to add a comment within your code
morekeywords={*},									% if you want to add more keywords to the set
keywordstyle=\bfseries\color{Blue1},
commentstyle=\itshape\color{Red4},
stringstyle=\itshape\color{Green4},
}


\newcommand{\includecpp}[2]{
  \subsection{#1}
    \lstinputlisting{#2}
}

\newcommand{\includetex}[2]{
  \subsection{#1}
    \input{#2}
}


\begin{document}
  \begin{multicols}{4}
  \pagestyle{fancy}
  
  \fancyfoot{}
  \fancyhead[L]{National Tsing Hua University - Duracell}
  \fancyhead[R]{\thepage}
  
  \renewcommand{\headrulewidth}{0.4pt}
  \renewcommand{\contentsname}{Contents}

   
  \scriptsize
  \section{VIM}
  \includecpp{vimrc}{./VIM/vimrc.cpp}
\section{data\_structure}
  \includecpp{Sparse\_Table}{./data_structure/Sparse_Table.cpp}
  \includecpp{segment\_Tree}{./data_structure/segment_Tree.cpp}
  \includecpp{disjointset}{./data_structure/disjointset.cpp}
\section{geometry}
  \includecpp{closest\_point}{./geometry/closest_point.cpp}
  \includecpp{points}{./geometry/points.cpp}
  \includecpp{lines}{./geometry/lines.cpp}
  \includecpp{geometry\_template}{./geometry/geometry_template.cpp}
  \includecpp{cp\_geometry.}{./geometry/cp_geometry..cpp}
\section{geometry/Convex\_Hull}
  \includecpp{Andrew's\_Monotone\_Chain}{./geometry/Convex_Hull/Andrew's_Monotone_Chain.cpp}
\section{graph}
  \includecpp{Kosaraju\_for\_SCC}{./graph/Kosaraju_for_SCC.cpp}
  \includecpp{Tarjan\_for\_BridgeCC}{./graph/Tarjan_for_BridgeCC.cpp}
  \includecpp{Tarjan\_for\_AP\_Bridge}{./graph/Tarjan_for_AP_Bridge.cpp}
  \includecpp{Tarjan\_for\_SCC}{./graph/Tarjan_for_SCC.cpp}
\section{graph/Bipartite}
  \includecpp{konig\_algorithm}{./graph/Bipartite/konig_algorithm.cpp}
  \includecpp{Kuhn-Munkres}{./graph/Bipartite/Kuhn-Munkres.cpp}
\section{graph/Flow}
  \includecpp{Ford\_Fulkerson}{./graph/Flow/Ford_Fulkerson.cpp}
  \includecpp{Edmonds-Karp-adjmax}{./graph/Flow/Edmonds-Karp-adjmax.cpp}
  \includecpp{Dinic\_algorithm}{./graph/Flow/Dinic_algorithm.cpp}
  \includecpp{Edmonds\_Karp\_2}{./graph/Flow/Edmonds_Karp_2.cpp}
  \includecpp{MinCostMaxFlow}{./graph/Flow/MinCostMaxFlow.cpp}
\section{graph/Matching}
  \includecpp{blossom\_matching}{./graph/Matching/blossom_matching.cpp}
\section{graph/Minimum\_Spanning\_Tree}
  \includecpp{prim}{./graph/Minimum_Spanning_Tree/prim.cpp}
  \includecpp{Kruskal}{./graph/Minimum_Spanning_Tree/Kruskal.cpp}
\section{graph/Shortest\_Path}
  \includecpp{dijkstra}{./graph/Shortest_Path/dijkstra.cpp}
  \includecpp{dijkstra-alrightchiu-version}{./graph/Shortest_Path/dijkstra-alrightchiu-version.cpp}
  \includecpp{bellman-Ford}{./graph/Shortest_Path/bellman-Ford.cpp}
  \includecpp{Floyd-Warshall}{./graph/Shortest_Path/Floyd-Warshall.cpp}
  \includecpp{shortest-path\_on\_DAG}{./graph/Shortest_Path/shortest-path_on_DAG.cpp}
\section{graph/Tree}
  \includecpp{Lowest\_Common\_Ancestor}{./graph/Tree/Lowest_Common_Ancestor.cpp}
  \includecpp{Tree\_Centroid}{./graph/Tree/Tree_Centroid.cpp}
\section{hashing}
  \includecpp{hashingVec}{./hashing/hashingVec.cpp}
\section{number\_theory}
  \includecpp{Fib}{./number_theory/Fib.cpp}
  \includecpp{BigInterger}{./number_theory/BigInterger.cpp}
  \includecpp{gcds}{./number_theory/gcds.cpp}
  \includecpp{nCr}{./number_theory/nCr.cpp}
\section{string}
  \includecpp{KMP}{./string/KMP.cpp}

  \clearpage
  \end{multicols}
  \newpage
  \begin{multicols}{4}
  \enlargethispage*{\baselineskip}
  \begin{center}
    \Huge\textsc{ACM ICPC Team Reference - Duracell!}
    \vspace{0.35cm}    
  \end{center}
  \tableofcontents
  \end{multicols}
  \clearpage
\end{document}