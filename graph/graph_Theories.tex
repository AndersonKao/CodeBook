\subsubsection{Definition}
\begin{itemize}
	\setlength\itemsep{-0.5em}
	\item Vertex Cover: Pick some vertices s.t. each edge covered by a least one vertex
	\item Matching : Pick some edge s.t. no two edge share same vertex.
	\item Indepentent vertex Set: Pick some vertices s.t. no two vertices are neighbor.
	\item edge(vertex) cactus: A graph every edge(vertex) belongs to at most one simple cycle.
\end{itemize}


\subsubsection{Konig's Theorem}

In any bipartite graph, the number of edges in a maximum matching equals the number of vertices in a minimum vertex cover.

\subsubsection{Constructing Minimum Cover}

二分圖左邊為 X集合 右邊為 Y集合,S和T分別代表從X的unmatch點開始,可以透過alternating path走到、各自屬於 X 和 Y 的點。$S\subseteq X, T\subseteq Y$
則 min cover  = $T \cup \left(X-S\right)$

\subsubsection{Independent Set on Bipartite graph}

In any bipartite graph, the complement of minimum vertex cover is a maximum Independent set.

\subsubsection{Minimum Weighted Vertex Cover}

二分圖的minimum weighted vertex cover可以透過最大流求出,建模方式如下:source 連向所有左邊的點,capacity是點權,所有右邊的點連向sink,capacity是點權,對於二分圖中原本有的邊,從左邊連向右邊,capacity為INF。可以透過此圖的 min cut構造出 vertex cover,而min cut可以透過此圖的 max flow 求出。

In bipartite graph, minimum weighted vertex cover = maximum weighted bipartite matching M.

\subsubsection{Hall's condition}

令左邊的點集合為$X$,右邊的點集合為$Y$,$|X| = |Y| = n$
If $|N(S)| \geq |S|$ for all $S \subseteq X$, then exist $n$ matching(perfect matching).
$N(S)$代表S的neighbor set


\subsubsection{Biconnected Component}
If a Biconnected component is 2-connected
\begin{enumerate}
	\setlength\itemsep{-0.5em}
	\item it has a least three vertices. (special case: 2 vertex and one edge)
	\item 表示任兩點間存在兩條互斥路徑(環)。且認兩邊可以找到一個環包含兩邊
	\item 若該連通分量有一個奇環則雙連通分量的任一點都至少一個奇環覆蓋
\end{enumerate}
\subsubsection{DFS Tree}
\begin{enumerate}
	\setlength\itemsep{-0.5em}
	\item The back-edges of the graph all connect a vertex with its descendant in the spanning tree.
	\item A back-edge is never a bridge. The edge between u and its parent is a bridge if and only if $\mathrm{dp}[u] = 0$ ( $\mathrm{dp}[u] = (\text{\# of back-edges going up from } u) - (\text{\# of back-edges going down from } u) + \underset{v \text{ is a child of } u}{\sum \mathrm{dp}[v]} )$.
	\item In DFS tree of a \emph{cactus}, for any span-edge, at most one back-edge passes over it. Each back-edge forms a simple cycle together with the span-edges it passes over. There's no other simple cycles. ( contract $\mathrm{cycleId}[v] != v \&\& \text{there is no back-edge going down from v}$ )
\end{enumerate}
\subsubsection{Euler Tour}
\begin{enumerate}
	\setlength\itemsep{-0.5em}
	\item An undirected graph has an Eulerian cycle iff $deg(u)$ is even for all $u$, and all vertices with $deg(u)!=0$ belong to a c.c..
	\item An undirected graph can be decomposed into edge-disjoint cycles iff all $deg(u)$ is even . So, a graph has an Eulerian cycle iff it can be decomposed into edge-disjoint cycles and its nonzero-degree vertices belong to a single c.c.
	\item An undirected graph has an Eulerian trail iff exactly zero or two vertices have odd degree, and all of its vertices with nonzero degree belong to a single c.c.
	\item A directed graph has an Eulerian cycle iff every vertex has equal in-deg and out-deg, and all vertices with nonzero degree belong to a single SCC. Equivalently, a directed graph has an Eulerian cycle iff it can be decomposed into edge-disjoint directed cycles and all with nonzero degree belong to a single SCC.
	\item A directed graph has an Eulerian trail iff at most one vertex has $(\text{out-deg}) − (\text{in-deg}) = 1$, at most one vertex has $(\text{in-deg}) − (\text{out-deg}) = 1$, every other vertex has equal in-degree and out-degree, and all vertices with nonzero degree belong to a single c.c. of the underlying undirected graph.
\end{enumerate}

